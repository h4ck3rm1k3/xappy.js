\documentclass{beamer}
%packages
\usepackage[latin1]{inputenc}
\usepackage{graphicx}
\usepackage{hyperref}
\usepackage[english]{babel}

\hypersetup{urlcolor=red,colorlinks}
\definecolor{bg}{rgb}{0.95,0.95,0.95}
\usetheme{Rochester}

\title{Sprint 2}
\author{xapi-team}
\institute{Institute for Computer Science, Free University Berlin}
\date{26.6.2011}
\begin{document}

\begin{frame}
\titlepage
\end{frame}

\begin{frame}{Parser}
    \begin{block}{Features}
        \begin{itemize}
            \item Nodes
            \item Ways
            \item Relations
            \item All Elements
            \item Bounding Box
            \item Tags
            \item Child Predicates
            \item Excaping
        \end{itemize}
    \end{block}
\end{frame}

\begin{frame}{Parser 2}
    \begin{block}{Input}
        \begin{itemize}
            \item http://server/api/0.6/node
            \item http://server/api/0.6/way
            \item http://server/api/0.6/relation
            \item http://server/api/0.6/*
            \item http://server/api/0.6/*[bbox=Int,Int,Int,Int]
            \item http://server/api/0.6/*[highway=motorway|trunk]
            \item http://server/api/0.6/relation[not(relation)]
        \end{itemize}
    \end{block}
\end{frame}

\begin{frame}{Validator}

    \begin{block}{Validates}
        \begin{itemize}
            \item bbox out of range
            \item bbox: left and right or top and bottom are swapped
            \item tag predicate with child predicate no(tags)
            \item tag predicate with child predicate tags
            \item node object with child predicate attribute node
            \item way object with child predicate attribute way
            \item way object and child predicate attribute node
        \end{itemize}
    \end{block}

\end{frame}

\begin{frame}{xapi Request Object}

    \begin{block}{object}
        \begin{itemize}
            \item node
            \item way
            \item relation
            \item *
        \end{itemize}
    \end{block}

    \begin{block}{bbox*}
        \begin{itemize}
            \item left
            \item right
            \item top
            \item bottom
        \end{itemize}
    \end{block}

\end{frame}

\begin{frame}{xapi Request Object (2)}
    \begin{block}{tag*}
        \begin{itemize}
            \item key
            \item value
        \end{itemize}
    \end{block}

     \begin{block}{child*}
        \begin{itemize}
            \item has
            \item attribute: 
                \begin{itemize}
                    \item node
                    \item way
                    \item relation
                    \item tag
                \end{itemize}
        \end{itemize}
    \end{block}


\end{frame}

\begin{frame}{XML Generator}
    \begin{itemize}
        \item genxml(type, elem)
        \item createHeader()
        \item createFooter()
    \end{itemize}

    \begin{block}{The header}
        $<$ ?xml version = "1.0" standalone="no"?$>$ \\
        $<$ osm \\
        version = "0.6" \\
        generator="xappy.js v0.2" \\
        xapi:planetDate = "200803150826"\\
        xapi:copyright = "2011 OpenStreetMap contributors"$>$\\
        \dots\\
        $</$osm$>$
    \end{block}
\end{frame}

\begin{frame}{JSON Generator}
    \begin{itemize}
        \item genjson(type, elem)
        \item createHeader()
        \item createFooter()
    \end{itemize}

    \begin{block}{The header}
        \{\\
            \hspace*{4mm}"version": 0.6,\\
            \hspace*{4mm} "generator": "xappy.js v0.2",\\
            \hspace*{4mm} "xapi":\{\\
            \hspace*{8mm} "planetDate": 200803150826,\\
            \hspace*{8mm} "copyright": "XXX"\\
            \hspace*{4mm} \},\\

            \hspace*{4mm}"elements": [\\
            \hspace*{8mm}\dots\\
            \hspace*{4mm}]\\
            \}
    \end{block}
\end{frame}





\begin{frame}{Internal Data Structure}
    \begin{block}{Node}
    \end{block}
    \begin{block}{Way}
    \end{block}
    \begin{block}{Relation}
    \end{block}
\end{frame}

\begin{frame}{Xappy.js Modules}
    This module contains glue code to stick various elements of the module to gether.

    Therefore various helpers are used for:

    \begin{itemize}
        \item Reading a config file
        \item Command line interface
        \item Loading all modules
        \item Configure a Callback for node.js
    \end{itemize}
\end{frame}

\begin{frame}{Validator}

    \begin{block}{Validates}
        \begin{itemize}
            \item bbox out of range
            \item bbox: left and right or top and bottom are swapped
            \item tag predicate with child predicate no(tags)
            \item tag predicate with child predicate tags
            \item node object with child predicate attribute node
            \item way object with child predicate attribute way
            \item way object and child predicate attribute node
        \end{itemize}
    \end{block}

\end{frame}

\begin{frame}{postgresdb - Queries}
    \begin{block}{Queries}
        \begin{itemize}
            \item Prepared statements
            \item name, sqlstatement, values
            \item secure, fast
        \end{itemize}
    \end{block}
\end{frame}
\begin{frame}{postgresdb - QueryPlan}
    \begin{block}{QueryPlan}
        \begin{itemize}
            \item Groups queries for all tables
            \item One query for every table (nodes, ways, relations)
        \end{itemize}
    \end{block}
\end{frame}
\begin{frame}{postgresdb - QueryBuilder}
    \begin{block}{QueryBuilder}
        \begin{itemize}
            \item creates queries and groups them into QueryPlans based on xapiRequests
        \end{itemize}
    \end{block}
\end{frame}
\begin{frame}{postgresdb - main}
    \begin{block}{main}
        \begin{itemize}
            \item passes xapiRequestsObject to queryBuilder
        \end{itemize}
    \end{block}
\end{frame}

\begin{frame}{EventHandler}

    Nessecary because of async behaviour of node.

    \begin{itemize}
        \item callbacks must be registered for events
        \item pushes objects from DB
        \item events emitted befor first and last
    \end{itemize}

    The EventHandler is created by DB the module, callbacks are defined in xappy.js.

\end{frame}

\begin{frame} {whosh - a new approach to osm data import}
    \begin{block}{What?}
        \begin{itemize}
        \item tool to import osm geo data into a postgres database
        \item push data into the database as fast as possible
        \item actions that are not possible while streaming data are handled by the db
        \end{itemize}
    \end{block}
    \begin{block}{Why?}
        \begin{itemize}
        \item all available tools need a lot of time for the import
        \item all available tools need a lot of ram
        \item most available tools do more than we need
        \end{itemize}
    \end{block}
\end{frame}

\begin{frame} {whosh - a new approach to osm data import}
    \begin{block}{How?}
        \begin{itemize}
        \item we use osmium - a library for parsing osm data in multiple formats
        \item library executes callbacks on node, way or relation
        \item independent COPY command for nodes, way and relations
        \item COPY allows to insert data fast (and unchecked) into tables
        \item functions in pg/SQL create geometry objects
        \item streaming and using pg/SQL should solve ram and time issues
        \end{itemize}
    \end{block}    
\end{frame}

\begin{frame} {whosh - a new approach to osm data import}
    \begin{block}{Future?}
        \begin{itemize}
        \item currently text format for COPY \\
        $\rightarrow$ binary format would be better
        \item speed and usability benchmarks
        \item world domination ... eh import
        \end{itemize}
    \end{block}    
\end{frame}

\begin{frame}{Coding Style}
    \normalsize
    Combination of Google \& Crockford\\
    Enforcement with the tool JSHint
\end{frame}
\end{document}
